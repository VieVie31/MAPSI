\documentclass{article}
\usepackage[francais]{babel}
\usepackage[utf8]{inputenc}
\usepackage[T1]{fontenc}
\usepackage{amssymb}
\usepackage{amsmath}

\title{Chapitre 2}
\author{VieVie31}

\begin{document}

\newtheorem{theo}{Théorème}
\newtheorem{coro}{Corollaire}

\maketitle

\section{Loi de Bernoulli}
Epreuve de Bernoulli : expérience aléatoire qui ne peut prendre que 2 résultats : {succès, échec}
\begin{itemize}
    \item $p$ : probas de succès
    \item $q = 1 - p$ : probas d'échec
    \item $X$ : le nombre de succès de l'épreuve de Bernoulli
    \item $E(X) = p$
    \item $V(X) = p(1 - p)$
\end{itemize}

\section{Loi Binomiale}
Epreuve binomiale : 
\begin{itemize}
    \item on répète $n$ fois la même épreuve de Bernoulli
    \item les probas $p$ et $q$ restent inchangées pour chaque épreuvede Bernoulli
    \item les épreuve de Bernoulli sont toutes réalisées indépendament les unes des autres
    \item $n$ : nombre de fois que l'on répète l'épreuve de Bernoulli
    \item $p$ : proba d'un succès
    \item $X$ : nombre de succès de l'épreuve Binomiale
    \item $X \sim B(n, p)$ : "$X$ suit une loi binomiale de paramètres $n$, $p$"
    \item $P(X=k) = {{n}\choose{k}}\ p^k(1 - p)^{n-k}, \quad \forall k=0,...,n$
    \item $E(X) = np$
    \item $V(X) = np(1-p)$
\end{itemize}

\section{Loi Normale}
\begin{itemize}
    \item $X \sim \mathcal{N} (m,\sigma^2)$
    \item $f(x) = \frac{1}{\sqrt{2 \pi} \sigma} \exp(-\frac{1}{2}(\frac{x - \mu}{\sigma})^2)$
    \item $E(x) = \mu$
    \item $V(x) = \sigma^2$
\end{itemize}

\begin{theo}
    Si $X \sim \mathcal{N}(m, \sigma^2)$ alors la variable $Y = aX + b$ obéit à la loi $\mathcal{N}(a\mu+b, a^2\sigma^2)$
\end{theo}

\begin{coro}
    Si $X \sim \mathcal{N}(\mu, \sigma^2)$ alors $Z = \frac{X - \mu}{\sigma}$ suit la loi $\mathcal{N}(0, 1)$
\end{coro}

\end{document}

