\documentclass{article}
\usepackage[francais]{babel}
\usepackage[utf8]{inputenc}
\usepackage[T1]{fontenc}
\usepackage{amssymb}
\usepackage{amsmath}

\title{Chapitre 5 \\ Tests Statistiques}
\author{VieVie31}

\begin{document}

\newtheorem{theo}{Théorème}
\newtheorem{coro}{Corollaire}

\newcommand{\argmax}{\operatornamewithlimits{argmax}}
\newcommand{\likelihood}{{\cal L}}

\maketitle

\section{Test d'Hypothèse}

On pose comme hypothèse $H_0$ l'hypothèse dont on est le plus sûr.

Soit :

\begin{itemize}
    \item $n$ : le nombre de valeurs de l'échantillon
    \item $\mu$ : moyenne théorique (ne pas confondre avec la moyenne de l'échantillon)
    \item $\bar X$: la moyenne empirique
    \item $\sigma$ : l'écart type de la distribution habituelle
    \item $\alpha$ : le niveau de confiance est de $1 - \alpha$
    \item $\bar X$ : limite supérieure / inférieure pour que l'hypothèse $H_0$ soit considérée comme vraie (à trouver)
\end{itemize}

On sait que :

\begin{align*}
    \frac{\bar X - \mu_{de\ l'hypothese}}{\sigma / \sqrt{n}} \sim \mathcal{N}(0, 1)
\end{align*}

Ainsi on cherche dans la table la valeur de $Z_{\alpha}$ tel que $P(X \geq Z_{\alpha}) = \alpha$.\\
Puis trouve la valeur de $\bar X = Z_{\alpha} * (\sigma / \sqrt{n}) \pm \mu$ et compare le $\bar X$ trouvé avec le réel.

\section{Force du Test}

On cherche le seuil qui va minimiser $\beta = P(H_1 rejet | H_1 vraie)$ soit maximiser $1 - \beta = P(H_0 rejet | H_1 vraie)$.

\section{Test d'Ajustement}

TODO

\section{Test de $\chi2$}

Soit une table de contingence $N$, on cherche à montrer que l'hypothèse $H_0$ (les deux variables sont indépendantes) est vraie avec un taux \textbf{d'erreur} de $\alpha$. \\
On commence par contruire la table $N$ construite à partir des marginales $L_i$, $C_j$ de $N$ dans son hypothèse d'indépendance. \\

\begin{align*}
    I_{i,j} = \frac{L_i * C_j}{n}
\end{align*}

On mesure ensuite la valeur de $d^2$ : 

\begin{align*}
    d^2 = \sum\limits_i \sum\limits_j \frac{(N_{i,j} - I_{i,j})^2}{I_{i,j}}
\end{align*}

On cherche ensuite dans la table la valeur de seuil en fonction de $\alpha$ et du nombre de degré de liberté calculé comme suit : $(\#L - 1) * (\#C - 1)$.\\
Enfin on vérifie que $d^2 < \chi^2$, si c'est le cas, il y a indépendance. 

\end{document}

