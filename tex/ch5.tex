\documentclass{article}
\usepackage[francais]{babel}
\usepackage[utf8]{inputenc}
\usepackage[T1]{fontenc}
\usepackage{amssymb}
\usepackage{amsmath}

\title{Chapitre 5 \\ Tests Statistiques}
\author{VieVie31}

\begin{document}

\newtheorem{theo}{Théorème}
\newtheorem{coro}{Corollaire}

\newcommand{\argmax}{\operatornamewithlimits{argmax}}
\newcommand{\likelihood}{{\cal L}}

\maketitle

\section{Test d'Hypothèse}


\section{Force du Test}


\section{Test de $\chi2$}

Soit une table de contingence $N$, on cherche à montrer que l'hypothèse $H_0$ (les deux variables sont indépendantes) est vraie avec un taux d'erreur de $\alpha$. \\
On commence par contruire la table $N$ construite à partir des marginales $L_i$, $C_j$ de $N$ dans son hypothèse d'indépendance. \\

\begin{align*}
    I_{i,j} = \frac{L_i * C_j}{n}
\end{align*}

On mesure ensuite la valeur de $d^2$ : 

\begin{align*}
    d^2 = \sum\limits_i \sum\limits_j \frac{N_{i,j} * I_{i,j}}{I_{i,j}}
\end{align*}

On cherche ensuite dans la table la valeur de seuil en fonction de $\alpha$ et du nombre de degré de liberté calculé comme suit : $(\#L - 1) * (\#C - 1)$.

\end{document}

