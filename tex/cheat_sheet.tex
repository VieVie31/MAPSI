\documentclass{article}
\usepackage[francais]{babel}
\usepackage[utf8]{inputenc}
\usepackage[T1]{fontenc}
\usepackage{amssymb}
\usepackage{amsmath}

\title{Formules Usuelles}
\author{VieVie31}

\begin{document}

\newtheorem{theo}{Théorème}
\newtheorem{coro}{Corollaire}

\maketitle

\section{Dérivées}

\begin{equation}
    (f \circ g)' = g' \times f' \circ g
\end{equation}

\begin{equation}
    \Big( \frac{1}{x} \Big)' = \frac{-1}{x^2} 
\end{equation}

\begin{equation}
    (\ln x)' = \frac{1}{x}
\end{equation}

\begin{equation}
    \Big( \frac{u}{v} \Big)' = \frac{{u'v} – {uv'}}{v^2}
\end{equation}

\section{Calculs}

\begin{equation}
    \frac{-1}{1 - p} = \frac{1}{p - 1}
\end{equation}

\begin{equation}
    \ln ab = \ln a + \ln b    
\end{equation}

\begin{equation}
    \ln a^b = b \ln a
\end{equation}



\section{Produits et Sommes}

\begin{equation}
    \prod x = \exp \sum \ln x
    \iff \sum x = \ln \prod \exp x
\end{equation}

\begin{equation}
    \sum \limits^n c = nc
\end{equation}

\begin{equation}
    \sum \limits^n (a_i - 1) = \Big( \sum \limits^n a_i \Big) -n
\end{equation}

\end{document}

