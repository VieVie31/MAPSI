\documentclass{article}
\usepackage[francais]{babel}
\usepackage[utf8]{inputenc}
\usepackage[T1]{fontenc}
\usepackage{amssymb}
\usepackage{amsmath}

\title{Chapitre 3}
\author{VieVie31}

\begin{document}

\newtheorem{theo}{Théorème}
\newtheorem{coro}{Corollaire}

\newcommand{\argmax}{\operatornamewithlimits{argmax}}
\newcommand{\likelihood}{{\cal L}}

\maketitle


\section{Vraisemblance d'un échantillon (loi discrète connue)}
$\likelihood(X) = $ vraisemblance de l'échantillon \textit{sachant la loi $P$}.

\begin{itemize}
    \item échantillon $\textbf{x} = (x_1, ..., x_n)$ de taille $n$
    \item échantillon : les $x_i = $ réalisation de variables aléatoires $X_i$
    \item les $X_i$ sont mutuellement indépendants (hypothèse essentielle)
\end{itemize}

\begin{equation}
    \likelihood(X) = P(x_1, ..., x_n) = \prod\limits_{i=1}^n P(X_i = x_i)
\end{equation} 


\section{Estimation par Maximum de Vraisemblance}

\begin{itemize}
    \item paramètre à estimer : $\Theta$
\end{itemize}

Calcul de vraisemblance de l'échantillon :
\begin{equation}
    \likelihood(x, \theta) = P(x_1, ..., x_n | \Theta=\theta) = \prod\limits_{i=1}^n P(x | \Theta = \theta)
\end{equation}

On cherche à maximiser $\theta$:
\begin{equation}
    \argmax\limits_{\theta \in \Theta} \likelihood(x, \Theta)
    \iff \argmax\limits_{\theta \in \Theta} \ln \likelihood(x, \Theta) 
    \iff \argmax\limits_{\theta \in \Theta} \sum\limits_{i=1}^n\ln P(x_i | \theta)
\end{equation}

Qui est obtenu losrque (sous conditaion de concavité et de dérivabilité):
\begin{equation}
    \frac{\partial \likelihood(x, \theta)}{\partial \theta} = 0
    \iff \frac{\partial \ln\likelihood(x, \theta)}{\partial \theta} = 0
    \iff \sum\limits_{i=1}^n \frac{\partial \ln \likelihood(x, \theta)}{\partial \theta} = 0
\end{equation}


\subsection{Calcul de l'Estimation de Vraisemblance}
Données : 
\begin{itemize}
    \item loi de probabilité $P(X = n) = f(x)$
    \item échantillon $(x_1, ..., x_n)$
\end{itemize}
Poser :
\begin{equation}
    \prod\limits^n P(X = n) = \prod\limits^n f(x)
\end{equation}
Puis calculer son logarithme :
\begin{equation}
    \sum\limits^n \ln P(X = n) = \sum\limits^n \ln f(x)
\end{equation}
Puis calculer sa dérivée partielle en fonction du paramètre $\theta$ et trouver où elle s'annule:
\begin{equation}
    \frac{\partial f(x)}{\partial \theta} = 0
\end{equation}

\end{document}

