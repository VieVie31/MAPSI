\documentclass{article}
\usepackage[francais]{babel}
\usepackage[utf8]{inputenc}
\usepackage[T1]{fontenc}
\usepackage{amssymb}
\usepackage{amsmath}

\title{Chapitre 1}
\author{VieVie31}

\begin{document}

\newtheorem{theo}{Théorème}
\newtheorem{coro}{Corollaire}

\maketitle


\section{Espérance}
L'espérance se définie comme suit :
\begin{equation}
    E(X) = \sum x_kp_k
\end{equation}

\begin{itemize}
    \item $E(aX + b) = aE(X) + b$
    \item $E(X + Y) = E(X) + E(Y)$
\end{itemize}



\section{Variance}
La variance se définie comme suit :
\begin{equation}
    V(X) = \sigma^2 = \sum[x_k-E(X)]^2p_k = E([X - E(X)]^2)
\end{equation}

L'écart type $\sigma$ est la racine carrée de la variance.

\begin{itemize}
    \item $V(aX + b) = a^2V(X)$
    \item $V(X) = E[X^2] - E[X]^2$
    \item $V(X+Y) = V(X) + V(Y) + 2\ cov(X, Y)$
\end{itemize}



\section{Covariance}
\begin{equation}
    cov(X, Y) = \sum[x_k - E(X)][y_k - E(Y)] p_k
\end{equation}

\begin{itemize}
    \item $V(X+Y) = V(X) + V(Y) + 2\ cov(X, Y)$
    \item $cov(X, Y) = E[XY] - E[X]E[Y]$
\end{itemize}



\section{Loi Marginale}
Permet de projetter une loi jointe sur l'une des variables aléatoires.\\
eg. extraire $P(A) = \sum\limits_i P(A, B=b_i)$



\section{Probabilités Conditionnelles}
\begin{equation}
    P(A | B) = \frac{P(A,B)}{P(B)} \iff P(A, B) = P(A | B) P(B)
\end{equation}

\begin{equation}
    P(A | B) P(B) = P(B | A) P(A) = P(A, B)
\end{equation}


\section{Indépendance}
$A, B$ sont indépendants si :
\begin{itemize}
    \item $P(A, B) = P(A)P(B)$
    \item $P(A | B) = P(A)$
\end{itemize}


\section{Coefficient de Corrélation Linéaire}

\begin{equation}
    r = \frac{cov(X, Y)}{\sigma_x\sigma_y}
\end{equation}

\begin{itemize}
    \item $r \in [-1, 1]$
    \item si $|r|$ est proche de 1 les variables sont dépendantes
\end{itemize}

\end{document}

